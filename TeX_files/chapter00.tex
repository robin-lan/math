\chapter{前言}
\paragraph{}作者只接受赞美,不接受批评;很久没翘尾巴了,来吧,赞美我吧。
\paragraph{}写这本书是因为国内没有专门从自然数一步一步推导到积分的书,大部分都把这部分内容,分散到微积分或者数学分析的各个章节,一带而过。既然没人做,那么就由我来做吧。
\paragraph{}应该由我来做,因为写书的人不能太聪明,恰好我就是那种被取外号“大地瓜”的人。大部分的人都是普通的智商,注意力不集中而且不爱思考,看聪明人写的书是一种痛苦,乱七八糟一堆下来最后得出一个定理、公式。什么鬼嘛!这样的书,给我这种“地瓜”看,还不如发个手册给我查呢。
\paragraph{}对呀,有手册就够了,可是为什么还让学呢?对照着公式,代入就可以了啊。这里边肯定有问题,也许写的人就不想让你懂,也许教的人不想让你学。当然,这也许是无心之举。毕竟当打开搜索引擎,出来的多半是娱乐新闻;打开视频平台,出来的多半是卖笑;看产品发布会,吹嘘着罔顾人命的商品。这个追逐资本,揠苗助长,剩下不多良心的时代,还能有什么更高的期望;写书的人被晋升、科研鞭打,教书的人被考分、评比压榨。还剩下多少时间、机会可以好好的写书,教出独立思考的学生。
\paragraph{}学的目的还剩下什么,“用”是一个。代入公式,跟着别人的经验走,解决问题。然而当AI能做这些时,听说狗狗也能数数,那我们比AI、狗狗强在那呢?我想可能是“思考”。
\paragraph{}一直以来,我都是娱乐至死的,认为思考是个很累的事,哪有刷短视频来得爽。然而当我看到有人写代码超过4个if、for的嵌套,goto满天飞,内存、进程、线程、文件概念不清,不明白linker和执行文件格式的关系时,我有种深深的独孤求败的感觉。原来思考能让人更容易的发现问题,解决问题。
\paragraph{}我是个臭写代码的程序员,大部分时候是在逆向、调试别人的程序。这是一份充满乐趣的工作,因为我的工作就是在思考,别人是怎么达到目的的。当逆向一段代码,在没有理解完一段代码后,我根本不知道对方是如何做到的。这时候。大概我是知道对方是要做什么,然而如何在一片茫茫的代码中,定位到重要信息却是个问题。解决这个问题的入口点是:如果我是他,我会怎么做。做这些肯定会留下痕迹,从这些痕迹中,一步一步还原对方的代码。在这个过程中,你会看到,对方时而无奈(为了限制条件,一直在跳转),时而粗心(怎么到这就完啦)。也能看到上古留下的痕迹(上古机器内存有限,避免递归,工作提到外层做循环)。
\paragraph{}原来“思考”是如此让人开心。这个过程就像看故事,感受到别人的开心、沮丧、困惑;又像是读一部历史,从小缝隙窥探秘密。

{\itshape
	\paragraph{}开写第5天的感慨:我就是一没苦硬吃的缺心眼、傻缺。整整写了一星期,还只写到0,1都还没开始数。我在这跪求不写书的聪明人原谅。写点东西,太难了,我理解你们了,我错了。
	\paragraph{}写在第20天的感慨:写着写着就发现,这个需要补上结论,那个需要补上定义,千疮百孔不忍直视。写之前的还想着这笔记是壶好茶,值得品;现在觉得是一副药,有病才喝。
}
\paragraph{}如果你看到这,我就放心了。说明我前面,又菜又狂,一眼看去精神有点不正常的话还是有效果的。我尽力把我想说的写明白,期望获得您的指正,谢谢。