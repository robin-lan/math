\chapter{实数}
\subsection{发现无理数}
\paragraph{}接上回,比例数都可以化为有限的连分数,会不会有一种数不能用有限的连分数表示出来呢?有的!历史上因为这个问题,还发生了血案。
\paragraph{}虽然我不吝言辞地赞美古希腊文明是现代文明的孵化器,然而伴随着权威的树立,恶也如影随形。古希腊的毕达哥拉斯学派就是当时的一个权威,这个学派在数学领域认为万物是比例数。它在数学最杰出的成就是证明了勾股定理,并广为传播。这个学派里有个刺头,希帕索斯,他用勾股定理刺破了万物皆数的“政治正确”。
\paragraph{}当时的场景是这样的。希帕索斯怀揣着理想、梦想加入了毕达哥拉斯学派。学派也把宗门绝学“毕达哥拉斯定理”倾囊相授。
\paragraph{}{ \itshape 直角三角形的两条直角边长度分别为 \(a\) 和 \(b\),斜边长度为 \(c\),则三边满足:
	\[
	\boxed{a \times a + b\times b = c \times c}
	\]
}
\paragraph{}这时希帕索斯就盯着这个绝学开始琢磨点东西,拿上这把屠龙刀开始笔画起来:直角边都是 1 的直角三角形。根据定理:
\begin{align*}
	1 \times 1 + 1 \times 1 &= c \times c \\
	2 &= c \times c
\end{align*}
根据万物皆数,c可以是$\frac{p}{q}$,其中p和q必须是一个奇数、一个偶数或者都是奇数。不然的话,就是还能化简。那么
\begin{align*}
	2 &= \frac{p \times p}{q \times q} \\
	2 \times q \times q &= p \times p
\end{align*}
根据偶数的定义,2乘以任何自然数结果是偶数,所以p $\times$ p 应该是偶数。再根据奇数乘以奇数必然是奇数,而自然数除了奇数就是偶数,所以p应该是偶数。所以p可以2 $\times$ k。
\begin{align*}
	2 \times q \times q &= 2 \times k \times 2 \times k \\
	q \times q &= 2 \times k \times k
\end{align*}
诶,也就是q也应该是偶数。可是在设定c是$\frac{p}{q}$的时候就约定了p和q不能都是偶数,出现了矛盾。不是万物皆数错了,就是勾股定理错了。
于是希帕索斯把这个发现告诉毕达哥拉斯的弟子们。
\paragraph{}
\begin{tabular}{@{} r p{0.8\textwidth} @{}}
	希帕索斯:& 直角边都是1的直角三角形,它的斜边不是数。真的。 \\
	弟子们:& 你懂个屁!毕达哥拉斯学派是民族之光,迷信派卡脖子的时候就它敢硬刚,承认万物皆数是支持毕达哥拉斯,你这种只会盯挑刺,怕不是收了迷信派的钱吧? \\
	希帕索斯:& 我就事论事说证明,你帮我看看,证明是哪里有错误吗? \\
	弟子们:& 迷信派好,你让迷信派帮你看证明。它们连毕达哥拉斯定理都没有,你怎么不说。 \\
	希帕索斯:& 你就帮我看看,是不是还有可以改进的地方,或者我没意识到的地方。 \\
	弟子们:& 你那么厉害,你搞个定理,开个学派。你想想是谁第一个证明这个定理的,要有大局观,别给敌对势力递刀子! \\
	希帕索斯:& 来,我们就说这个证明,好吧。那条斜边是不是\dots\dots \\
	弟子们:& 谁逼你证明直角边是1的三角形了,有矛盾不会自己消化!迷信派有这技术吗?也就你们这种人,享受着毕达哥拉斯的红利还天天骂娘!
\end{tabular}
\paragraph{}经过弟子们的一通商议:解决不了问题,还解决不了希帕索斯吗。乘着月高风黑,直接把希帕索斯绑了,然后一脚踢到海里,给沉了。回来的路上还骂:先有毕哥后有天。希帕索斯,该(gay)。
\paragraph{}希帕索斯帮助我们找到并证明了一个不是比例数的数,这个数是画在直角三角形上的,那它到底是多少呢?我们用连分数描述它:
\begin{align}
	c \times c &= 2  & \text{} \label{sqr2:1}\\
	c \times c - 1 &= 1 & \text{\ref{sqr2:1}两边减1} \label{sqr2:2}\\
	c \times c - c + c - 1 &= 1 & \text{\ref{sqr2:2}左边加c减c} \label{sqr2:3} \\
	(c + 1) \times (c - 1) &= 1 & \text{\ref{sqr2:3}左边乘法分配律} \label{sqr2:4}\\
	c - 1 &= \frac{1}{c + 1} & \text{\ref{sqr2:4}两边同除c+1且c$\neq$-1} \label{sqr2:5}\\
	c &= 1 + \frac{1}{1 + c} & \text{\ref{sqr2:5}两边同减1}
\end{align}
把c化简得到
\[
	\boxed{c = 1 + \frac{1}{1 + c}}
\]
把c代入到这个式子迭代看看:
\paragraph{}
\begin{align*}
	c &= 1 + \cfrac{1}{1 + \left(1 + \cfrac{1}{1 + \left(1 + \frac{1}{1 + c}\right)}\right)} \\
	  &= 1 + \cfrac{1}{2 + \cfrac{1}{2 + \cfrac{1}{1 + c}}} \\
	  &= 1 + \cfrac{1}{2 + \cfrac{1}{2 + \cfrac{1}{2 + \ddots}}} 
\end{align*}
\paragraph{}所以c = \([1; 2, 2, \dot{2}]\)。
\begin{theorem}
	无限连分数的每次渐进展开计算不断靠近某个数,奇数次渐进连分数递增,偶数次渐进连分数递减。
	\[
	\cfrac{p_1}{q_1}<\cfrac{p_3}{q_3}<\cdots<\cfrac{p_{n+1}}{q_{n+1}}<\cdots < x < \cdots<\cfrac{p_n}{q_n}<\cdots<\cfrac{p_2}{q_2}<\cfrac{p_0}{q_0}
	\]
\end{theorem}
\begin{Proof}
	设
	{\itshape $x_{n}$是无限连分数的余项}
	\begin{align}
		x_n &= a_n + \cfrac{1}{a_{n+1} + \cfrac{1}{a_{n+2} + \cfrac{1}{\ddots}}} &\text{} \label{cf:pc1}\\
		x_n &= a_n + \cfrac{1}{x_{n+1}} &\text{} \label{cf:pc2} \\
		x_n &> a_n &\text{根据\ref{cf:pc2}} \label{cf:pc3} 
	\end{align}
	根据余项的定义可得:
	\begin{align}
		x_{n+1} &=  a_{n+1} + \cfrac{1}{a_{n+2} + \cfrac{1}{\ddots}} &\text{} \label{cf:pc4} \\
		x_{n+1} &> a_{n+1} &\text{根据\ref{cf:pc4}} \label{cf:pc5} \\
		\cfrac{1}{x_{n+1}} &< \cfrac{1}{a_{n+1}}  &\text{根据\ref{cf:pc5}} \label{cf:pc6} \\
		x_{n} &< a_{n} + \cfrac{1}{a_{n+1}} &\text{根据\ref{cf:pc2}\ref{cf:pc6}} \label{cf:pc7}
	\end{align}
	根据\ref{cf:pc3}和\ref{cf:pc7}可得:
	\begin{align}
		a_{n} < x_{n} < a_{n} + \cfrac{1}{a_{n+1}} \label{cf:pc8}
	\end{align}
	把$a_{n}$和$a_{n} +\cfrac{1}{a_{n+1}}$代入
\begin{align}
	\cfrac{p_n}{q_n} = \cfrac{a_n \cdot p_{n-1} + p_{n-2}}{a_n \cdot q_{n-1} + q_{n-2}} \label{cf:cp9}
\end{align}
分别得到:
\begin{align}
	\ref{cf:cp9} &\Rightarrow f(a_n) = \cfrac{p_n}{q_n} \label{cf:cp10}\\
	\ref{cf:cp9} &\Rightarrow f(a_{n} + \cfrac{1}{a_n+1}) = \cfrac{p_{n+1}}{p_{n+1}} \label{cf:cp11}
\end{align}
根据上个章节的连分数渐进分数的	奇数次渐进连分数递增,偶数次渐进连分数递减。\\
	当n是奇数时,渐进连分数是根据变量递增的数,并且根据\ref{cf:pc8} \ref{cf:cp10} \ref{cf:cp11}:
	\[
	a_n < x_n < a_n + \cfrac{1}{a_{n+1}}  \Rightarrow \cfrac{p_n}{q_n} < f(x_n) < \cfrac{p_{n+1}}{q_{n+1}} 
	\]
	当n是偶数时,渐进连分数是根据变量递减的数,并且根据\ref{cf:pc8} \ref{cf:cp10} \ref{cf:cp11}:
	\[
	a_n < x_n < a_n + \cfrac{1}{a_{n+1}}  \Rightarrow \cfrac{p_n}{q_n} > f(x_n) > \cfrac{p_{n+1}}{q_{n+1}} 
	\]
	n属于自然数,只有奇数和偶数两种形式。不管奇数还是偶数,无限连分数的渐进展开都位于$\cfrac{p_n}{q_n}$和$\cfrac{p_{n+1}}{q_{n+1}}$之间,而且
	根据上一章有限连分数的结论:
	\[
	\cfrac{p_1}{q_1}<\cfrac{p_3}{q_3}<\cdots<\cfrac{p_{n+1}}{q_{n+1}}<\cdots \leq x(\text{比例数}) \leq \cdots<\cfrac{p_n}{q_n}<\cdots<\cfrac{p_2}{q_2}<\cfrac{p_0}{q_0}
	\]
	所以无限连分数渐进计算之间的距离不断缩小,奇数次渐进连分数递增,偶数次渐进连分数递减。
\end{Proof}
由于是无限连分数,所以它不像有限连分数,最终会等于一个比例数值;无限连分数依赖目前的比例数概念,它不会等于某个比例数数,等于的话它就是有限连分数了。
\[
\cfrac{p_1}{q_1}<\cfrac{p_3}{q_3}<\cdots<\cfrac{p_{n+1}}{q_{n+1}}<\cdots < x(\text{非比例数}) < \cdots<\cfrac{p_n}{q_n}<\cdots<\cfrac{p_2}{q_2}<\cfrac{p_0}{q_0}
\]
通过分析上面的不等式,我们发现无限连分数的渐进连分数被一个特别的数分割为两个部分:左边没有最大比例数。理由是如果有一个$\cfrac{p_{n+1}}{q_{n+}}$,根据自然数没有最大值,还有$\cfrac{p_{n+3}}{q_{n+3}}$,$\cfrac{p_{n+1}}{q_{n+1}} < \cfrac{p_{n+3}}{q_{n+3}}$。同样的,右边没有最小比例数;而且这个特别的数不是比例数。

\begin{definision}\label{def:partition}
	分割:把一个非空集合分成两个非空集合,分成的两个集合中的元素不重复,合并两个集合的元素等于被分割的集合。
\end{definision}
因为比例数的稠密性,所以我们可以把大于或者小于x(非比例数)的渐进连分数(比例数)扩大到所有比例数去定义无理数。
\begin{definision}\label{def:Irration}
	一个特定的非比例数数把比例数分割为两部分,其中一部分没有最大比例数,另一部分没有最小比例数。定义这个数为无理数。
\end{definision}
\subsection{戴德金分割}
回顾自然数、整数、比例数的创建和表示过程:自然数利用与集合的基(元素个数)等价的方式建立,整数利用自然数加法等价的方式建立,比例数利用整数乘法等价的方式建立,实数呢?很显然,利用需要用到比例数。
\paragraph{}就像无理数的定义,无理数把比例数分割成了两个部分,其中一部分是没有最大比例数,另一部分是没有最小比例数,但把这两部分的所有比例数合并起来,得到的是所有的比例数。这遵循\textbf{定义\ref{def:partition}--分割的定义}:分割得出的集合非空、并集为全集、交集为空集。
\begin{dv_convension}\label{dv:1}
	比例数分割成A、B两个集合。对任意a $\in$ A、b $\in$ B都有a $<$ b。
\end{dv_convension}
\paragraph{}把比例数集分割成两部分,有以下四种情况:
\begin{itemize}
	\item 集合A\textbf{有}最大比例数,集合B\textbf{有}最小比例数。
	\item 集合A\textbf{有}最大比例数,集合B\textbf{没有}最小比例数。
	\item 集合A\textbf{没有}最大比例数,集合B\textbf{有}最小比例数。
	\item 集合A\textbf{没有}最大比例数,集合B\textbf{没有}最小比例数。
\end{itemize}
\paragraph{}如果分割的数是比例数集合,那么\textit{“集合A\textbf{有}最大比例数,集合B\textbf{有}最小比例数。”}这种情况是不可能出现的。
\begin{Proof}
	反证法:
	\begin{itemize}[label=]
		\item 比例数A集合最大的数是a;
		\item 比例数B集合最小的数是b。
	\end{itemize}
	根据\textbf{推论}\ref{cor:ration1}:比例数的加、减、乘、除还是比例数。
	\begin{align}
		& c = \cfrac{a + b}{2} = a + \cfrac{b - a}{2} = b - \cfrac{b - a}{2} \quad c \in Q \label{pf:errsplit0_1} \\
		& c = a + \cfrac{b - a}{2} \quad \Rightarrow \quad c > a \label{pf:errsplit0_2} \\
		& c = b - \cfrac{b - a}{2} \quad \Rightarrow \quad c < b \label{pf:errsplit0_3}
	\end{align}
	根据\ref{pf:errsplit0_2}和\textit{“比例数A集合最大的数是a”},c不是A集合的元素;根据\ref{pf:errsplit0_3}和\textit{“比例数B集合最小的数是b”},c不是B集合的元素。根据\textbf{定义\ref{def:partition}}--分割的定义:分割得出的集和的并集为全集。比例数A集合和比例数B集合的并集不是全体比例数,因为还遗漏了c这个比例数。
\end{Proof}
\begin{itemize}
	\item 集合A\textbf{有}最大比例数,集合B\textbf{没有}最小比例数。
	\item 集合A\textbf{没有}最大比例数,集合B\textbf{有}最小比例数。
	\item 集合A\textbf{没有}最大比例数,集合B\textbf{没有}最小比例数。
\end{itemize}
\paragraph{}分割的是比例数,\textit{“集合A\textbf{有}最大比例数,集合B\textbf{没有}最小比例数”}和\textit{“集合A\textbf{没有}最大比例数,集合B\textbf{有}最小比例数”}都可以唯一地确定一个比例数。比如比例数2:
\begin{align}
	\text{比例数集合A\textbf{有}最大比例数2,比例数集合B\textbf{没有}最小比例数。} \label{re:split_ration1}\\
	\text{比例数集合A\textbf{没有}最大比例数,比例数集合B\textbf{有}最小比例数2。} \label{re:split_reation2}
\end{align}
\paragraph{}分割的是比例数,根据无理数的\textbf{定义\ref{def:Irration}},\textit{“集合A\textbf{没有}最大比例数,集合B\textbf{没有}最小比例数”},确定了一个无理数。
\begin{corollary}\label{co:split_q2r}
	比例数分割唯一地确认了一个实数。
\end{corollary}
\paragraph{}因为\textit{“集合A\textbf{有}最大比例数,集合B\textbf{没有}最小比例数”}和\textit{“集合A\textbf{没有}最大比例数,集合B\textbf{有}最小比例数”}都可以唯一地确定一个比例数。为了便于讨论,只取一种兼容确定无理数的分割\textit{“集合A\textbf{没有}最大比例数,集合B\textbf{没有}最小比例数”}表达方式,所以做出以下约定:
\begin{dv_convension}\label{dv:2}
	 “集合A没有最大比例数”的切割为合法的切割。
\end{dv_convension}
\begin{definision}\label{df:proper_subset}
	真子集是指一个集合的所有元素都属于另一个集合,但两个集合并不完全相等(即后者至少包含一个前者没有的元素)。A是B的真子集:
	\[
	A \subsetneqq B \iff (\forall x(x \in A \implies x \in B)) \land (\exists y(y \in B \land y \notin A))
	\]
\end{definision}
\paragraph{}分割确认了一个实数点,评估这个分割点的办法可以是比较分割后的集合里的比例数的集合。设$a_0$、$a_1$是两个分割点,$A0_{q}$、$B0_{q}$是以$a_0$为分割点,分割比例数的集合;$A1_{q}$、$B1_{q}$是以$a_1$为分割点,分割比例数的集合;其中集合A的比例数小于集合B的比例数。
\begin{definision}
	根据分割约定:实数点a把比例数分割为两个非空集合A、B,集合A的所有比例数都小于集合B任意的比例数,且集合A没有最大比例数。
	\begin{itemize}[label=$\circ$]
		\item 完备性:$\forall$x $\in$ $\mathbb{Q}$(x $\in$ A $\lor$ x $\in$ B) $\iff$ A $\cup$ B = $\mathbb{Q}$
		\item 非空性:$\exists$x $\in$ $\mathbb{Q}$(x $\in$ A) $\land$ $\exists$y $\in$ $\mathbb{Q}$(y $\in$ B)
		\item 互斥性:$\neg$($\exists$x $\in$ $\mathbb{Q}$(x $\in$ A $\land$ x $\in$ B))
		\item 有序性:$\forall$x $\in$ A $\forall$y $\in$ B(x $<$ y)
		\item 下集无最大比例数:$\forall$x $\in$ A $\exists$$x'$ $\in$ A(x$<$$x'$)
	\end{itemize}
\end{definision}
\paragraph{}完备性由\textbf{分割定义\ref{def:partition}}(并集是全局)得出;非空性也是由\textbf{分割定义\ref{def:partition}}(分割成两个非空集)得出;互斥性同样由\textbf{分割定义\ref{def:partition}}(不交)得出;有序性由\textbf{分割约定\ref{dv:1}}得出;下集无最大比例数由\textbf{分割约定\ref{dv:2}}得出。
\begin{definision}\label{def:realex}
	一个实数在比例数分割约定的条件下,等价集合$A_{q}$或集合$B_{q}$。
\end{definision}
\paragraph{}\textbf{定义\ref{def:realex}}的目的是说,单个实数可以用比例数分割后的集合来表示。当谈论一个实数时,可以认为谈论的是一个比例数集合。
\begin{definision}
	实数是比例数和无理数的统称。实数集是比例数集和无理数集的并集。实数集用符号$\mathbb{R}$表示。
\end{definision}
\subsection{有界实数集的确界}
\paragraph{}下集无最大比例数的逻辑符号($\forall$x $\in$ A $\exists$$x'$ $\in$ A(x$<$$x'$)),这样读试试:所有在集合A的数,都存在另一个在集合A的数比它(所有在集合A的数)大。明显有问题:既然是所有的元素了,怎么还能有另一个,感觉是自己比自己大?所以严格的来说$\forall$应该是“任意”,而不是通常语言的“所有”;任意是从个体推广到所有,严格逻辑博弈的视角。例如,\{1,2,3,4\}的所有元素都小于5,\{1,2,3,4\}中任何一个元素都小于5,这两句话都是正确的;但我们要进行证明时,有限的集合可以枚举所有,但无限的集合采用“任意”这条路会更加的简单,方便。
\paragraph{}$\forall$x $\in$ A $\exists$$x'$ $\in$ A(x$<$$x'$)是2步骤博弈,博弈是真是假由x$<$$x'$决定。博弈的目的是让自己的利益最大化,逻辑博弈中最终的命题语句前的逻辑词是一方。在这个逻辑语句里,x$<$$x'$前的$\exists$是一方,另一方就是$\forall$,$\exists$目的是x$<$$x'$为真。首先由$\forall$一方出牌,拿出集合A中的任意一个数x;然后由另一方$\exists$在A中找出是否存在一个数$x'$;如果x$<$$x'$成立,那么$\exists$一方获胜,整个命题成立。
\paragraph{}尝试把$\forall$的命题转为否定形式。例如:所有的鸟都会飞,$\forall$ 鸟(会飞)。这句话的否定是:至少存在一只鸟它不会飞,$\exists$ 鸟(不会飞)。$\forall$的否定规则:
\[
\boxed{\neg(\forall x P(x)) = \exists x (\neg P(x))}
\]
\paragraph{}再次尝试把 P$\iff$Q 转为否定形式。P$\iff$Q 的逻辑含义是它们同时真或者同时假,也就是 (P$\land$Q) $\lor$ ($\neg$P $\land$ $\neg$Q)。同真同假的否定是一真一假,一真一假是 ($\neg$P$\land$Q) $\lor$ (P$\land$$\neg$Q)。P$\iff$Q的否定规则:
\[
\boxed{\neg(P \iff P) = (\neg P \land Q)\lor(P \land \neg Q)}
\]
\begin{definision}\label{df:real_order}
	两个实数间的序关系:
\begin{itemize}
	\item $a_0 = a_1 \iff A0_q = A1_q \iff \forall x (x \in A0_q \iff x \in A1_q)$
	\item $a_0 < a_1 \iff A0_q  \subsetneqq A1_q \iff \exists x (x \in A0_q \land x \notin A1_q)$
	\item $a_0 > a_1 \iff A0_q  \supsetneqq A1_q \iff \exists x (x \in A1_q \land x \notin A0_q)$
\end{itemize}
\end{definision}
\begin{corollary}\label{co:real_order}
	两个实数序的关系,有且只有以下一种关系成立:等于(=)、小于($<$)、大于($>$)。
\end{corollary}
\begin{Proof}
	需要证明“有”、“只有一种关系”
	\begin{itemize}[label=$\circ$]
		\item 有(至少有一种关系存在)\\
		反证法证明,如果能证明$a_0 = a_1$、$a_0 < a_1$、$a_0 > a_1$不能同时成立,则至少有一种关系是成立的。\\ 
		假设$a_0 = a_1$不成立,那么需要否定$\forall x (x \in A0_q \iff x \in A1_q)$。根据$\forall$的否定规则:
		\[
		\neg(\forall x(x \in A0_q \iff x \in A1_q)) = \exists x (\neg(x \in A0_q \iff x \in A1_q)))
		\]
		再根据P $\iff$ Q的否定规则:
		\[
		\exists x (\neg(x \in A0_q \iff x \in A1_q)) = \exists x((x \in A0_q \land x \notin A1_q) \lor (x \notin A0_q \land x \in A1_q))
		\]
		$\exists x((x \in A0_q \land x \notin A1_q) \lor (x \notin A0_q \land x \in A1_q))$可分解为
		\begin{align}
			[\exists x(x \in A0_q \land x \notin A1_q)]
			\lor [\exists x(x \notin A0_q \land x \in A1_q)] \label{re:s1}
		\end{align}	
		总结:如果$a_0 = a_1$不成立为真,那么\ref{re:s1}也为真。然而\ref{re:s1}要为真,则需
		\begin{align}
			\exists x(x \in A0_q \land x \notin A1_q) \label{re:s2} \\
			\exists x(x \notin A0_q \land x \in A1_q) \label{re:s3}
		\end{align}
		其中之一为真就可以。如果\ref{re:s2}为真,则$a_0 < a_1$;如果\ref{re:s3}为真,则\text{$a_0 > a_1$}。\\
		所以$a_0 = a_1$、$a_0 < a_1$、$a_0 > a_1$不能同时不成立
		\item 只有一种关系(不可能多种关系)\\
			1. $a_0 = a_1 \iff \forall x (x \in A0_q \iff x \in A1_q)$,即对任意一个x,如果属于$A0_q$,就属于$A1_q$。\\
			2. $a_0 < a_1 \iff \exists x (x \in A0_q \land x \notin A1_q)$,即存在x,属于$A0_q$,但不属于$A1_q$。\\
			3. $a_0 > a_1 \iff \exists x (x \in A1_q \land x \notin A0_q)$,即存在x,属于$A1_q$,但不属于$A0_q$。\\
			\begin{itemize}[label=]
				\item $a_0 = a_1$显然是不能和$a_0 < a_1$、$a_0 > a_1$共存的。因为$a_0 = a_1$是任意一个x在$A0_q$,也在$A1_q$;在$A1_q$,也在$A0_q$。而$a_0 < a_1$存在x在$A0_q$,但不在$A1_q$;$a_0 > a_1$存在x在$A1_q$,但不在$A0_q$。
				\item 那么$a_0 < a_1$和$a_0 > a_1$是否可以共存呢?\\
				当 
				\[
				a_0 < a_1 \iff \exists x_0 (x_0 \in A0_q \land x_0 \notin A1_q)
				\]
				由完备性$A_q \cup B_q = \mathbb{Q}$,$x_0 \notin A1_q$可推导得出$x_0 \in B1_q$,则:
				\begin{align}
				a_0 < a_1 \iff \exists x_0 (x_0 \in A0_q \land x_0 \in B1_q)	\label{re:s4}			
				\end{align}
				当 
				\[
				a_0 > a_1 \iff \exists x_1 (x_1 \in A1_q \land x_1 \notin A0_q)
				\]
				由完备性$A_q \cup B_q = \mathbb{Q}$,$x_1 \notin A0_q$可推导得出$x_1 \in B0_q$,则:
				\begin{align}
				a_0 > a_1 \iff \exists x_1 (x_1 \in A1_q \land x_1 \in B0_q)	\label{re:s5}			
				\end{align}
				由\ref{re:s4}\ref{re:s5}:
				\[
					\exists x_0 \exists x_1(x_0 \in A0_q \land x_1 \in B0_q \land  x_1 \in A1_q \land x_0 \in B1_q)
				\]
				结合比例数的有序性:
				\[
				\forall x \in A \forall y \in B(x < y)
				\]
				得到:
				\begin{align}
					(x_0 < x_1) \land (x_1 < x_0) \label{re:s6}
				\end{align}
				由于$x_0$、$x_1$是比例数,根据两个比例数关系=、$<$、$>$有且只有一种情况\ref{rat:rat2},所以\ref{re:s6}矛盾,也就是$a_0 < a_1$和$a_0 > a_1$不能同时存在。
			\end{itemize}
	\end{itemize}
\end{Proof}
\paragraph{}设E是一个实数集合。
\begin{definision}
	如果E的每个元素,都小于或者等于实数$\alpha$,则称$\alpha$为集合E的\textbf{上界}。
	\[
		\forall x \in E, x \leq \alpha
	\]
	根据实数序的定义:x $\leq$ $\alpha$,则$X_q \subseteq A_q$。$\alpha$是E集合的上界,等价表示为:
	\[
		\forall x \in E, X_q \subseteq A_q
	\]
\end{definision}
\begin{definision}
	如果$\alpha$是E的上界,并且任何的实数$\beta < \alpha$,$\beta$都不是E的上界。则称$\alpha$是E的\textbf{上确界},也称为\textbf{最小上界},记作$\alpha$ = $\sup$E。
	\[
		\alpha = \sup E \Leftrightarrow (\forall x \in E, x \leq \alpha) \land (\forall \beta \in \mathbb{R}, \beta < \alpha \Rightarrow \exists x_0 \in E, x_0 > \beta)
	\]
	根据实数的序关系,上确界可以表达为:
	\[
		\alpha = \sup E \Leftrightarrow (\forall x \in E, X_q \subseteq A_q) \land (\forall \beta \in \mathbb{R}, B_q \subsetneqq A_q \Rightarrow \exists x_0 \in E, B_q \subsetneqq X0_q)
	\]
\end{definision}
\begin{corollary}
	非空有上界的实数集必有上确界
\end{corollary}
\begin{Proof}
	首先根据戴德金分割,符号描述非空有上界的实数集:
	\begin{itemize}[label=$\circ$]
		\item S: 非空有上界的实数集。
		\item $A_\alpha$: 根据\textbf{定义\ref{def:realex}}实数$\alpha$所对应的“没有最大比例数”的集合。
		\item $\bigcup_{\alpha \in S} A_\alpha$: 集合S中所有元素“没有最大比例数”的集合的并集。
	\end{itemize}
	 \[
	 A^* = \bigcup_{\alpha \in S} A_\alpha
	 \]
	 \indent 如果能证明$A^*$是戴德金分割的“没有最大比例数”的集合;再根据\textbf{推论\ref{co:split_q2r}}戴德金比例数分割唯一地确定了一个实数$\sigma$;最后根据\textbf{推论\ref{co:real_order}}实数的序关系可以得出$\sigma$是这个有上界实数集的上确界。\\
	 \indent 首先证明$A^*$匹配分割的定义:
	 \begin{itemize}[label=$\circ$]
	 	\item 非空: 根据命题,S是非空的实数集,所以存在一个属于S的实数$\alpha$。根据\textbf{定义\ref{def:realex}},$\alpha$等价的$A_\alpha$非空。那么$A_\alpha$的并集$A^*$也非空。
	 	\item 非全集: 根据命题,S是有上界的实数。根据上界的定义,存在一个实数$\beta$,所有属于S的实数都小于等于$\beta$。根据实数序的定义,所有属于S的实数等价对应的$A_q$都是$A_\beta$的子集。由于$A_\beta$不是比例数的全集,那么$A_q$的并集$A^*$也不会是比例数全集。
	 	\item 并集为全集:属于$A^*$的比例数是一个集合,那么可以让不属于$A^*$的比例数是另一个集合$A^{other}$。这两个集合的并集是比例数的全集。
	 \end{itemize}
	 \indent 其次证明$A^*$和$A^{other}$符合\textbf{比例数分割约定\ref{dv:1}}:对任何的$a \in A^*$、$b \in A^{other}$都有$a < b$。
	 \begin{itemize}[label=$\circ$]
	 	\item 因为$A^*$是集合S的实数对应的“没有最大比例数”的集合的并集,所以任何一个属于$A^*$的比例数a,必然在集合S中存在一个“没有最大比例数”的集合$A_a$,这个集合包含了比例数a。这里把这个集合对应的实数设为$\alpha'$。根据实数的分割约定\ref{dv:2}:$a < \alpha'$。
	 	\item 因为任意$A^{other}$中的比例数b,不属于任何$A^*$。等价地b都不属于集合S中的实数对应的“没有最大比例数”的集合。根据实数的分割约定\ref{dv:2},b与S中的实数的关系必然不是小于的关系。再根据实数的序关系:b只会是会大于等于集合S中任何的实数。
	 \end{itemize}
	 \indent 结合前面推导:\\
	 \indent $
	 \begin{cases}
	 	\forall a \in A^*, \exists \alpha' \in S (a < \alpha') \\
	 	\forall b \in A^{other}, \forall \alpha' \in S (\alpha' \leq b)
	 \end{cases}
	 \implies a < \alpha' \leq b \implies a < b $\\ \\
	 最后证明$A^*$符合\textbf{比例数分割约定\ref{dv:2}}:集合$A^*$没有最大比例数。\\
	 \indent 反证法,假设在$A^*$里存在一个最大的比例数$\beta$,因为$A^*$是$A_a$集合的并,所以必然存在一个$A_a$包含$\beta$。而$A_a$等价的是一个实数,这个实数是符合\textbf{比例数分割约定\ref{dv:2}},也就是说$A_a$没有最大比例数,那么也就是说$\beta$不会是包含$\beta$的“没有最大比例数”集合里最大的比例数。\\
	 \indent 根据上面的证明可以得出$A^*$是戴德金分割,并且根据\textbf{推论\ref{co:split_q2r}}:比例数分割唯一的确认了一个实数。那么存在一个实数s,是集合$A^*$所等价的实数。最后证明s是S这个实数集合的上确界。
	 \begin{itemize}[label=$\circ$]
	 	\item s是S集合的上界($\forall x \in S, X_q \subseteq A^*$)。$A^*$是集合S里所有实数等价的比例数分割类$X_q$的并,所以$X_q \subseteqq A^*$,根据\textbf{定义\ref{df:real_order}}两个实数间序的关系可以得出:
	 	\[
	 		\forall x \in S, x \leq s
	 	\]
	 	\item s是S集合的最小上界($\forall \beta \in \mathbb{R}, B_q \subsetneqq A^* \Rightarrow \exists x \in S, B_q \subsetneqq X_q$)。假设一个实数$\beta$小于s,根据实数序关系,得到$B_q \subsetneqq A^*$。因为$B_q$是$A^*$的真子集,根据\textbf{真子集定义\ref{df:proper_subset}},$A^*$集合中必然存在一个比例数x是$B_q$所没有的,也就是说必然存在一个包含比例数x的集合$X_x$($X_x \subsetneqq A^*$)。$X_x$对应的实数是x,那么
	 	\[
	 		\exists x \in S, x > \beta
	 	\]
	 	总结是:
	 	\[
	 		\forall \beta \in \mathbb{R}, B_q \subsetneqq A^* \Rightarrow \exists x \in S, B_q \subsetneqq X_q
	 	\]
	 \end{itemize}
\end{Proof}
\subsection{实数的四则运算}
根据两个实数的等于关系,可以定义实数间的加、减、乘、除:
\begin{definision}
	两个实数$\alpha$=($A_q$, $B_q$)、$\beta$=($C_q$, $D_q$)的加法:$\alpha$ + $\beta$ = ($E_q$, $F_q$)。
	\begin{align*}
		& E_q = \{a + c\ |\ a \in A \land c \in C\} \\
		& F_q = \mathbb{Q} \setminus E_q \quad	(\text{$\setminus$是集合的减法符号})
	\end{align*}
\end{definision}
\begin{definision}
	实数$\alpha$=($A_q$, $B_q$)的负数-$\alpha$=($C_q$, $D_q$)。
	\begin{align*}
		& \text{$\alpha$是有理数:}  &&\text{}\\
		& \text{}		&& C_q = \{-c\ |\ c \in B_q \land c \neq \alpha \} \\
		& \text{}		&& D_q = \mathbb{Q} \setminus C_q\text{是有理数}  \\
		& \text{$\alpha$是无理数:}  &&\text{}\\
		& \text{}		&& C_q = \{-c\ |\ c \in B_q \} \\
		& \text{}		&& D_q = \mathbb{Q} \setminus C_q
	\end{align*}
\end{definision}
\begin{definision}
	实数$\alpha$、$\beta$的减法:$\alpha$ - $\beta$ = $\alpha$ + (-$\beta$)
\end{definision}
\begin{definision}
	实数$\alpha$的绝对值$|\alpha|$:
	\begin{itemize}
		\item  $\alpha$ $>$ 0时,$|\alpha|$ = $\alpha$。
		\item  $\alpha$ $<$ 0时,$|\alpha|$ = -$\alpha$。 
	\end{itemize}
\end{definision}
\begin{definision}
	两个实数$\alpha$=($A_q$, $B_q$)、$\beta$=($C_q$, $D_q$)的乘法:
	\begin{itemize}
		\item $\alpha$ $>$ 0、$\beta$ $>$ 0:$\alpha$ $\times$ $\beta$ = ($E_q$, $F_q$)
		\begin{align*}
			& E_q = \{a \times b \ |\ (a \in A_q) a > 0 \land (b \in C_q) b > 0\} \cup \{x\ |\ (x \in \mathbb{Q}) x < 0\} \\
			& F_q = \mathbb{Q} \setminus E_q
		\end{align*}
		\item $\alpha$ $<$ 0、$\beta$ $>$ 0:
		\[
			\alpha \times \beta = -((-\alpha) \times \beta)
		\]
		\item $\alpha$ $>$ 0、$\beta$ $<$ 0:
		\[
		\alpha \times \beta = -(\alpha \times (-\beta))
		\]
		\item $\alpha$ $<$ 0、$\beta$ $<$ 0:
		\[
		\alpha \times \beta = (-\alpha) \times (-\beta))
		\]
	\end{itemize}
\end{definision}
\begin{definision}
	实数$\alpha$=($A_q$, $B_q$)的倒数$\cfrac{1}{\alpha}$=($C_q$, $D_q$)。
	\begin{itemize}
		\item $\alpha$ $>$ 0:
		\begin{align*}
			& C_q = \{\cfrac{1}{a}\ |\ (a \in B_q)a > 0\} \cup \{x\ |\ (x \in \mathbb{Q})x \leq 0\} \\
			& D_q = \mathbb{Q} \setminus C_q
		\end{align*}
		\item $\alpha$ $<$ 0:
		\[
		\cfrac{1}{\alpha} = -(\cfrac{1}{-\alpha})
		\]
	\end{itemize}
\end{definision}
\begin{definision}
	实数$\alpha$除以$\beta$:$\alpha$ $\times$ $\cfrac{1}{\beta}$
\end{definision}


