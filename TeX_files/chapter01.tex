\chapter{自然数}
\section{你认为的自然数}
\paragraph{}想了半天,还是先不把自然数的公理放出来,它过于抽象。如果你没办法想象它如何莫名其妙,那么我来告诉你。
\paragraph{}比如,我写了一个卖包子的程序,顾客来买包子,要几个包子程序就出几个包子。以严谨著称的升腾测试人员上场了:

她说:我要1千万个包子。

等了半天,程序没有反应。(其实程序正在努力的做包子中)

小本本记下:P2 bug, 很多包子买不了。

她继续说:好吧,那给我来2个包子,一个肉包,一个菜包。肉包里不要有肉,菜包里不要有菜。
等了一会,程序没有反应。

小本本继续记下:P1 bug, 2个包子也买不了。

她说:给我来0个包子。

程序:你不买就不买,来凑什么热闹,汗。

小本本最后记下:P0 bug,根本买不到包子。
\paragraph{}见识了上面测试人员的做法,就慢慢明白,为什么皮亚诺公理为什么要写得那么莫名其妙了。是的,做数学的人个个都是杠精,不整得无懈可击,会被抨得很惨。
\paragraph{}简单的说,如果让你来告诉我,什么是自然数,你会怎么告诉我?小朋友可能会说:先给我很多糖果,先有糖果后有数。如果是你,你可能会说:1, 2, 3, 4, \dots这样一直数,数出来的数就是自然数。然后再问下AI,发现我国高考遵循\textbf{国际标准 ISO 80000-2}和\textbf{国家基础教育教材}的规范自然数包含0,那就是从0开始数,一直数出来数是自然数。

说得没错,自然数就是你说的那些,而且只有那些,完全正确。
\section{再认识自然数}
\paragraph{}我现在化身杠精,想问问为什么从0开始数,数出来数是自然数,凭什么?老祖宗都说了:一生二,二生三,三生万物,从1数才是正解,才是自古以来。你只数到4,那5是不是自然数,9.9是不是自然数,很多是不是自然数,数不清是不是自然数......?总之,说从0数,数出来的数是自然数,我就是不服。
\subsection{从哪开始数}
\paragraph{}现在就0属不属于自然数,分为三个派别。第一个派别:传统派。认为0不属于自然数,这个派别应该是最古老的派别。你想,你数数会先数个0吗?会那么数的,C程序员无疑(让AI用C写个程序,分别打印字符串每个字符,你就知道为什么程序员是这么数的)。后来呢,像我这种懒人,偏听偏信的懒人多了,就发展出第二种派别:骑墙派。数数是从1开始数,那时候我是传统派。老师说高考定义是从0开始数,那时我就不是传统派。主打一个其乐融融,不惹事。
\paragraph{}那为什么把0加入到自然数,变成现在的正统了呢?那就要说到最后一个派别:集合论和逻辑学派。这个派别敬仰古希腊的荣光;就如同谈到西方政治,必然会有罗马帝国的痕迹。不论是希特勒第三帝国的建筑、艺术,还是美国政治人物的傲慢、自负,都似乎带着向罗马帝国致敬的味道,认为自己是伟大历史的书写者。古希腊的理性、思辨是人类的第一次觉醒,它塑造了现代文明:理性主义(不偏听偏信)、人文主义(生而平等)、公民意识(一切归劳动者所有,哪能容得寄生虫?!)。看到上面说的是不是热血沸腾,会不会感觉古希腊的荣光还不够亮。
\paragraph{}一帮做数学的,也年轻过,也中二过,他们认为开创历史不是梦。他们从古希腊继承了有一部秘籍《几何原本》。我们作为普通人生活在距离这本书2300年后的今天,生活中99\%的数学问题,都可以从中找到解决方法。这本书让人惊叹的地方是:它只依赖5 条公理(适用于所有学科的基本真理,如 “等于同量的量彼此相等”)和 5条几何公设(如“过两点能作且只能作一直线”“直角都相等”),也就是只要承认5条公里5条几何公设,通过严格的逻辑推导,必然能够证明 465 个命题,它涵盖平面几何、立体几何、数论等领域。那么集合论和逻辑学派的那帮人就思考,他们能不能也从几条基础的公理出发,根据逻辑推导最后重新建立数学大厦。
\paragraph{}初高中可能也学了集合,可是感觉集合也就那么回事,好像什么事也没发生。对了,这种感觉对了。集合如果不和逻辑发生关系,它什么都不是。
\paragraph{}集合是把一类东西放一起,就像放学排入队,一群学生站一起;一堆苹果放一个货架上。放一起的东西各式各样,那取个名字吧,叫元素。其实把元素叫东西也可以,但刚好集合的是一群人,叫东西就有些侮辱人了。干脆叫元素,反正你也没听说过,挑不出我毛病。
\paragraph{}看,多完美,我只是定义了一个框和东西,我们就可以开始数数了。等等,从哪里开始数呢,一开始集合里就有东西吗,有多少?数学家就想:从哪里开始数呢?就像《几何原本》它的公理只需要5条,那数学需要的基础也是越少越好,难怪说数学家有洁癖。专业的人一般对自己的专业会有洁癖,表现得专业。比如我就看不惯函数4层以上的嵌套,goto乱来。洁癖和犟种的区别是,犟种不是蠢就是自卑。洁癖的数学家说:干脆从什么都没有开始,多干净。这个意见获得了数学家的共识。于是一条公理被提了出来:存在不含任何元素的集合。也就是别吵了,从0开始数。
\subsection{数学语言}
\paragraph{}这里插播一下。大部分人上大学后会发现,大学的数学书看不懂了。那其实是,近代数学公理化之后,同行之间开始用黑话交流。

{\itshape
“天王盖地虎,宝塔镇河妖”:这是《林海雪原》中最经典的黑话对答,是杨子荣假扮土匪时与座山雕手下的接头暗号。
“天王盖地虎” 意为 “你好大的胆子,敢来欺辱你祖宗?”
“宝塔镇河妖” 意为 “我若撒谎,让我被河水淹死,受宝塔镇压。”
“西北玄天一朵云,乌鸦落在凤凰群”:形容自己人来到了厉害的团伙中,下句接 “满桌都是英雄汉,谁是君来谁是臣?” 用于试探地位。

这是现代互联网里的黑话:我们要从顶层设计出发,聚焦用户痛点,通过差异化策略和精细化的颗粒度运营,击穿用户心智,打造产品的核心竞争力。在这个过程中,要注重各个环节的拉通和对齐,形成完整的业务闭环。同时,以数据为抓手,进行抽离透传和归因分析,为决策提供有力支持,不断赋能业务发展,最终实现业务的持续增长,在行业中构建起我们的生态优势。
}

看到互联网黑话,我就倒胃恶心;能不能好好说话!用模糊、新概念掩盖自己空洞,掩盖自己对具体问题毫无处理方法。一句话就是:无能。数学是由最聪明的一群人推动的,它的黑话不是无能,而且为了应对杠精,从源头把傻缺给排除。所以没有刻意训练数学的语言,当然是看不懂大学的数学。好好学习这门语言吧,它会比英语更加通用,流传会更加悠远。更重要的是,学好后,你有资格和村口的大妈一决高下。

我们说:元素、集合

老外说:element、set

数学说:我用小写字母a、b、x等表示元素,用大写字母A、B、C等或者用\{\}表示集合;而空集如此特殊,被定义为这个符号:$\emptyset$。$\emptyset$是象形文字,用圆圈表示围起来的集合,用斜线划去表示什么都没有。例如一个包含a、b、c元素的集合A:\{a、b、c\}。
\begin{center}
	\begin{tcolorbox}[
		colback=white,
		colframe=blue!50!black,
		arc=3mm,
		boxrule=1pt,
		width=0.7\textwidth, % 框的宽度
		center, % 框内内容居中
		enlarge left by=0mm,
		enlarge right by=0mm,
		top=3mm,
		bottom=3mm,
		]
		空集公理:存在不含任何元素的集合。
		
		数学语言:$\exists$$\emptyset$$\forall$x(x $\notin$ $\emptyset$)
	\end{tcolorbox}
\end{center}
\paragraph{}上面的数学语言就是那批向古希腊致敬的数学家发明的,为了表达颠覆他们时代的意味,所以空集公理的数学语言里有两个符号是正常符号取反得来的。发现了吗:$\exists$、$\forall$。
\paragraph{}$\exists$左右旋转后是E,它源自拉丁语Existere的第一个字母。“Existere” 在拉丁语中由前缀 “ex-”(意为 “出、向外”)和词根 “sistere”(意为 “站立、存在”)组成,英语的exist(存在)就是这么来的。这个符号表的意思是:我就光明正大的站在这里。一般我们把他读作“存在”。存在就存在,有什么用呢?用处非常大,行走江湖抬杠必备。比如有人说:鸟都会飞。你说:存在一种鸟不会飞:鸵鸟。有人说:食物都会变馊。你说:存在一种食物不会变馊:蜂蜜。学会了吧,但凡别人说全部都怎么样,你只要举出$\exists$(存在)的一种反例,就能把他怼死。这就是$\exists$(存在)存在的意义。
\paragraph{}有认识$\forall$的朋友说:$\forall$是A的上下旋转,因为$\forall$的意思是“所有的”,那就是All的第一个字母。意思是对的,很可惜,来历不是这样的。它确实是所有的意思,但它是由拉丁语 “Omnis”(意为 “所有的、每一个”)的第一个字母得来的。我也无法理解$\forall$怎么就是O的旋转了,直到我看到o的手写体,为了连笔,o的左右俩边都有一个向下的线,上下旋转后变成向上的线,就变成了$\forall$。 $\forall$的目的是扩大打击面,就像那就经典台词:我不是针对你,我是说在座的各位$\forall$都是垃圾。又比如:你指出某些东西的不好,这时有人就说:\underline{任何}说**不好,$\forall$都是*黑。一般$\forall$可以用:“都”、“全部”、“任何一个、任意拿出一个”来表述,用“任何一个、任意拿出一个”表述最自然,最解决我们日常表达。
\paragraph{}上面的空集公理就还剩下最后一个符号了。$\notin$很明显那条斜线就是否定的意思,去掉斜线后是$\in$。$\in$源于拉丁语 “est”(某物是某集合的元素,即 “某物属于某集合”),这个单词和英语的“is”同一个意思。但我们不能简单的翻译成“是”,因为我们的“是”是等同的意思,而数学中,这个符号更准确的意思是“属于”、“是集合中的元素”。这个符号表示了元素和集合的关系。x$\in$A:x是A集合里的元素。x$\notin$A:x不是A集合里的元素。这里有个小细节,加上这个小细节,空集公理的数学语言我们就完全可以读懂了。这个细节是我们把可以改变的元素,用x、y、z这种小写字母表示,而用a、b、c这种小写字母表示不会改变的元素。
\paragraph{}重读空集公理:$\exists$$\emptyset$$\forall$x(x $\notin$$\emptyset$)。直译:存在空集和所有的元素,所有的元素都不属于空集。再译:存在空集和所有元素,从所有元素中取出任何元素,这个元素都不属于空集。说人话就是:存在空集,空集里不包含任何元素。
\paragraph{}好啦,废话一堆。目的就是说:从空集公理出发,我们的自然数从0开始数。空集里没元素,没有任何东西,我们就说空集表示0。
\subsection{我会数数了}
\paragraph{}有0了,那1是不是也要定义一个区别$\emptyset$的集合,叫1集合?为了遵循“如无必要,勿增实体”的原则,死犟死犟的数学家什么都没引入,他把空集再用了一遍,来表示1。我想到的是这种\{$\emptyset$\}。漂亮!这种表示1,无懈可击。接着来2,\{$\emptyset$、$\emptyset$\}。
\paragraph{}嗯......这个2的集合表示好像碰到了点问题。什么问题呢?举个例子:把家庭成员里,会打乒乓球的人组成一个集合,这个集合是A:\{我、弟弟\};把家庭成员里,是程序员的人组成一个集合,这个集合是B:\{我\}。现在把家庭成员里,会打乒乓球的人和程序员组成一个集合C。这个集合C是\{我、弟弟\},还是\{我、弟弟、我\}呢?很明显,看上去它们是有区别的,但实质上它们描述的是同一个事物。也就是集合中相同的元素即使重复多次,对集合的整体没有影响。
\paragraph{}那2到底要如何从上面仅有的元素、集合、空集的概念推导出来呢?$\emptyset$和$\emptyset$是同一个,然而$\emptyset$和\{$\emptyset$\}不是同一个。那2就可以用\{$\emptyset$、\{$\emptyset$\}\}表示。
\paragraph{}到这,我似乎发现了获得下一个集合的规律。把当前集合元素的所有元素合并到下一个集合,再加上一个“当前集合”的元素(集合作为元素)。
\begin{itemize}[label=]
	\item 0:$\emptyset$ $\rightarrow$ 把$\emptyset$作为下一个集合的元素
	\item 1:\{$\emptyset$\} $\rightarrow$ \underline{\{$\emptyset$\}}
	\item 2:\{$\emptyset$、\underline{\{$\emptyset$\}}\} $\rightarrow$ \underline{\{$\emptyset$、\{$\emptyset$\}\}}
	\item 3:\{$\emptyset$、\{$\emptyset$\}、\underline{\{$\emptyset$、\{$\emptyset$\}\}}\} $\rightarrow$ \underline{\{$\emptyset$、\{$\emptyset$\}、\{$\emptyset$、\{$\emptyset$\}\}\}}
\end{itemize}
\paragraph{}漂亮,后面的4、5、6\dots 不是问题了,按照规则数下去都能得到。 
\subsection{先填一个坑}
\paragraph{}首先要填的第一个坑就是:明明可以用一次加一个$\emptyset$的方式来创造自然数,也就是集合里有1个空集,表示1,两个空集表示2的这种方式来表示自然数。为什么舍近求远,你不能用“我是我,独一无二的我”这个口号来灌输我错误的观念,说得好听,我是被卖家秀骗过的人,我要解释!很遗憾,没法给出解释,真没办法给出解释。
\paragraph{}数学家对没办法解释的事物,处理办法相当简单粗暴。没办法解释是吧,那就提出一个公理:
\begin{center}
	\begin{tcolorbox}[
		colback=white,
		colframe=blue!50!black,
		arc=3mm,
		boxrule=1pt,
		width=0.9\textwidth, % 框的宽度
		center, % 框内内容居中
		enlarge left by=0mm,
		enlarge right by=0mm,
		top=3mm,
		bottom=3mm,
		]
		外延公理:两个集合的元素相等,那么这两个集合相等。
		
		数学语言:$\forall X \forall Y (\forall x (x \in X \iff x \in Y) \iff X = Y)$
	\end{tcolorbox}
\end{center}
\paragraph{}说实话,我一看到外延公理,我的第一反应是什么鬼。我说的不是,我不能理解外延公理的内容,而是“外延”这个词。“外延”是什么?能不能取个一目了然的名字:元素公理。等我带着这个问题,查看了“外延(extension)”、“内涵(intension)”的意思才知道,这两个词是一对,而且它们是哲学词汇。集合论的创立者“康托”在大学学习的是数学和哲学,那就难怪这公理的名字怪哲学,不是凡人可以明白。简单的说:“内涵”是我们的描述;“外延”是根据描述,能得到的东西。就例如:小于4的自然数是内涵,外延是\{0、1、2、3\}。从内涵你就能得出外延,但从外延不一定能得到你要的内涵。例如从外延\{0、1、2、3\},有人认为是小于4的自然数,有人认为是前4个自然数。使用内涵表示集合,甚至引发悖论,罗素悖论是个很有名的内涵悖论的例子。从上面的解释可以看出,要获得准确的、直接的信息,应该提取的是外延。在集合里,也就是元素。

{\itshape
有一个很有名的内涵悖论:罗素悖论。
	\begin{itemize}[label=$\bullet$]
		\item 某理发师宣称:“我只给所有不给自己理发的人理发。”	
		\item 请问:这个理发师该不该给自己理发。
		\begin{itemize}[label=$\circ$]
			\item 若他给自己理发,则违反 “不给自己理发” 的原则;
			\item 若他不给自己理发,则根据原则应给自己理发。
		\end{itemize}
	\end{itemize}
}
\paragraph{}接着来看看数学语言,有了上面学习数学语言的基础,外延公理只多出了一个两个符号:$\iff$、=。=是等于的意思。$\iff$是个两头都有箭头的符号,箭头代表推导,$\Rightarrow$(向右的箭头)表示左边的情况如果满足,那么右边必然也满足;同理$\Leftarrow$(向左的箭头)表示右边情况满足,左边必然也满足。重读外延公理:任意的两个集合A和B,任何的一个元素,只要这个元素在A里,那么这个元素必然也在B里;只要这个元素在B里,那么这个元素必然也在A里。那么我们说集合A等于集合B。倒回来,任意的两个集合A、B,A等于B,那么A的所有元素在B里,B的所有元素在A里。
\paragraph{}如果认为\{$\emptyset$\}代表1,\{$\emptyset$、$\emptyset$\}代表2,下面的证明推导出1=2的悖论。

\begin{proof}
	\{$\emptyset$\}只有1个元素$\emptyset$,$\emptyset$ $\in$ \{$\emptyset$、$\emptyset$\}。
	
	满足:$\forall$x(x $\in$ X $\Rightarrow$ x $\in$ Y)。
	
	
	\{$\emptyset$、$\emptyset$\}里有两个$\emptyset$,第一个$\emptyset$ $\in$ \{$\emptyset$\},第二个$\emptyset$ $\in$ \{$\emptyset$\}。
	
	满足:$\forall$x(x $\in$ X $\Leftarrow$ x $\in$ Y)。
	
	根据外延公理,得证\{$\emptyset$\} = \{$\emptyset$、$\emptyset$\}
\end{proof}
\paragraph{}所以根据配对公理,用\{$\emptyset$、$\emptyset$\}表示2不是可取的方法。
\subsection{再填一个坑}
\paragraph{}回顾上面说的创造下一个集合的规律:是把当前集合元素的所有元素合并到下一个集合,再加上一个“当前集合”的元素(集合作为元素)。

\begin{itemize}[label=]
	\item 0:$\emptyset$ $\rightarrow$ 把$\emptyset$作为下一个集合的元素。
	\item 1:\{$\emptyset$\} $\rightarrow$ \underline{\{$\emptyset$\}}
	\item 2:\{$\emptyset$、\underline{\{$\emptyset$\}}\} $\rightarrow$ \underline{\{$\emptyset$、\{$\emptyset$\}\}}
	\item 3:\{$\emptyset$、\{$\emptyset$\}、\underline{\{$\emptyset$、\{$\emptyset$\}\}}\} $\rightarrow$ \underline{\{$\emptyset$、\{$\emptyset$\}、\{$\emptyset$、\{$\emptyset$\}\}\}}
\end{itemize}
\paragraph{}首先要解决的第一个问题:两个集合必然能组成一个集合。这是我们从空集出发创造新集合做的第一个动作,我们需要把它合法化。遇事不决,出公理:
\begin{center}
	\begin{tcolorbox}[
		colback=white,
		colframe=blue!50!black,
		arc=3mm,
		boxrule=1pt,
		width=0.9\textwidth, % 框的宽度
		center, % 框内内容居中
		enlarge left by=0mm,
		enlarge right by=0mm,
		top=3mm,
		bottom=3mm,
		]
		配对公理:任意的两个集合a、b能组成新的集合C=\{a、b\}。
		数学语言:$\forall$a $\forall$b $\exists$C $\forall$x(x $\in$ C $\iff$ (x = a $\lor$ x = b))
	\end{tcolorbox}
\end{center}
\paragraph{}读数学语言:这句数学语言出现了新的符号:$\lor$,它读作“或”、“或者”。它的来源是拉丁语“vel”(或)的首写字母v。记不住的话,就把$\lor$当成一个坑,或者这个或者那个都可以往里边丢。它的用法是只要$\lor$的左右两边有一个有效就认为“$\lor$左右两边”这个整体有效。比如别人问你:什么时候踢进世界杯?你可以回答:草坪太干或者草坪太湿或者草坪不干不湿,我都会输球。说得输球好像不是你的问题,是草坪的问题。草坪那么多或者的情况,只要满足一条,你就可以输球。用数学的语言把配对公理读出来就是:任意两个集合(两个形成配对,所以叫配对公理),存在第三个集合,第三个集合里的任何一个元素是前面两个集合之一。
\paragraph{}配对公理只是解决了两个集合能组成第三个集合,而且第三个集合里的元素是前两个集合。我们通常把这种“集合的元素是集合”的集合称为集合族。其实应该叫集合群,也就是一群集合组成的集合,但群在数学已经被伽罗瓦提前占位了,族群、族群,只剩下族了,所以我们把这类集合叫集合族。虽然两个集合组成第三个集合已经合法化,但我们通常更加关注的是把前两个集合的元素打散放到第三个集合,也就是合并两个集合的元素。
\begin{center}
	\begin{tcolorbox}[
		colback=white,
		colframe=blue!50!black,
		arc=3mm,
		boxrule=1pt,
		width=0.9\textwidth, % 框的宽度
		center, % 框内内容居中
		enlarge left by=0mm,
		enlarge right by=0mm,
		top=3mm,
		bottom=3mm,
		]
		并集公理:存在一个集合,它的元素是集合族里集合的元素。
		
		数学语言:$\forall$X $\exists$Y $\forall$u(u $\in$ Y $\iff$ $\exists$z(z $\in$ X $\land$ u $\in$ z))
	\end{tcolorbox}
\end{center}
\paragraph{}读数学语言:这里又出现了一个数学符号$\land$,它刚好和我们从配对公理学的$\lor$方向相反。$\lor$是或者的意思,相反的$\land$是而且的意思。比如很多男人说的“肤白貌美大长腿”,是或者的意思,只要有一个满足就是美女;很多女人说的“有车有房”才结婚,是而且的意思,必须有车而且有房,都达到了才满足条件。严以律己、宽以待人说的是对自己要用而且,对别人要用或者。现在这句古训慢慢被另一句话替代:我也是第一次做人,凭什么?先通读一遍$\forall$X $\exists$Y $\forall$u(u $\in$ Y $\iff$ $\exists$z(z $\in$ X $\land$ u $\in$ z)),从z $\in$ X $\land$ u $\in$ z可以看出来,X是一个集合族,因为X的元素是z,而z的元素是u,也就是z是一个集合,X的元素是集合z。有了这些信息,从头开始读:任何的集合族,都存在一个集合,这个集合的元素是集合族里集合的元素。
\begin{definision}
	集合的并运算 X $\cup$ Y 定义为:
	\[
	X \cup Y = \{ x \mid x \in X \lor x \in Y \}
	\]
\end{definision}
让我们根据配对公理和并集公理,构造从0 $\rightarrow$ 1 $\rightarrow$ 2 $\rightarrow$ 3。
\begin{itemize}[label=]
	\item 0:$\emptyset$
	\item 1:\{$\emptyset$\} $\rightarrow$ \underline{\{$\emptyset$\}}
	\item 2:\{$\emptyset$、\underline{\{$\emptyset$\}}\} $\rightarrow$ \underline{\{$\emptyset$、\{$\emptyset$\}\}}
	\item 3:\{$\emptyset$、\{$\emptyset$\}、\underline{\{$\emptyset$、\{$\emptyset$\}\}}\} $\rightarrow$ \underline{\{$\emptyset$、\{$\emptyset$\}、\{$\emptyset$、\{$\emptyset$\}\}\}}
\end{itemize}
\paragraph{}构造从0 $\rightarrow$ 1。有两个集合$\emptyset$、\{$\emptyset$\},根据并集公理,必然有一个集合,这个集合的元素是两个集合的元素。$\emptyset$没有元素,\{$\emptyset$\}的元素是$\emptyset$,所以构造的集合是\{$\emptyset$\}。
\paragraph{}构造从1 $\rightarrow$ 2。有两个集合\{$\emptyset$\}、\{\{$\emptyset$\}\},根据并集公理,必然有一个集合,这个集合的元素是两个集合的元素,\{$\emptyset$\}集合的元素是$\emptyset$,\{\{$\emptyset$\}集合的元素是\{$\emptyset$\},所以构造的集合是\{$\emptyset$、\{$\emptyset$\}\}。
\paragraph{}构造从2 $\rightarrow$ 3。有两个集合\{$\emptyset$、\{$\emptyset$\}\}、\{\{$\emptyset$、\{$\emptyset$\}\}\},根据并集公理,必然有一个集合,这个集合的元素是两个集合的元素,\{$\emptyset$、\{$\emptyset$\}\}的元素是$\emptyset$、\{$\emptyset$\},\{\{$\emptyset$、\{$\emptyset$\}\}\}的元素是\{$\emptyset$、\{$\emptyset$\}\},所以构造的集合是\{$\emptyset$、\{$\emptyset$\}、\{$\emptyset$、\{$\emptyset$\}\}\}
\paragraph{}根据上面的方法,就可以一直的构造自然数。从现在开始,我宣布“你可以合法的数数了”。
\subsection{填最后的坑}
\paragraph{}从上面构造自然数,我们采用了一种方法构造下一个自然数的方法:$x^+ = x \cup \{x\}$。这种方法有个名称叫:归纳。我们在生活中还经常碰到一个词:递归。递归就像内卷,当整个社会产生了内卷的氛围,这种氛围就像病毒一样蔓延到每个公司,公司再把这个病毒传递给每员工,员工又在不知不觉中传递到家庭,又从家庭传递给每一个人。递归是一种简单的方法,但它能触及到任何的角落。作为程序员可以不知道归纳,但万万不能不知道递归。程序的函数是指把一个过程组合成一个整体可以操作的对象。递归是在这个函数过程中,再次执行了这个函数。例如你要获取计算机所有的文件名称,下面的几行递归伪代码实现了这个目的。
{\itshape
	\begin{itemize}[label=]
		\item 遍历目录(需要遍历的目录路径) \{
		\begin{itemize}[label=]
			\item 打开需要遍历的目录路径
			\item 查看当前层级的文件和目录
			\item 如果是文件,那么收集这个文件名称
			\item 如果是目录,那么再次执行\large{遍历目录(这个目录路径)}
		\end{itemize}
		\item \large{遍历目录(我的电脑)}
	\end{itemize}
}
\paragraph{}递归采用的是从大往小递进;归纳正好相反,它是从小往大延伸。仔细观察,生活中处处有递归和归纳,只要是一种能渗透进每个角落的传播方式,那它不是递归就是归纳,或者是两者震荡放大的结果。扶不扶摔倒的老人是归纳,买不买房是递归,内卷是从几家IT公司开始(You are shame)归纳到IT行业竞争氛围,然后通过氛围递归到整个社会。
\paragraph{}说了这么多,现在思考一个问题:归纳是一个过程,这个过程在自然数中是不会停止的,那需要怎么看待自然数。简单的说:把自然数看出一个过程,还是一个整体。也就是自然数是不是一个集合。小白会疑惑,这重要吗?当然,这关系到1 $\div$ 3 = 0.33$\dot{3}$ 里的0.33$\dot{3}$,你认为它是一个 真实存在的确切的数,还是一个过程。为了解决这个问题,再次提出了一个公理:
\begin{center}
	\begin{tcolorbox}[
		colback=white,
		colframe=blue!50!black,
		arc=3mm,
		boxrule=1pt,
		width=0.9\textwidth, % 框的宽度
		center, % 框内内容居中
		enlarge left by=0mm,
		enlarge right by=0mm,
		top=3mm,
		bottom=3mm,
		]
		无穷公理:存在由归纳产生的集合。
		
		数学语言:$\exists$S [$\emptyset$ $\in$ S $\land$ ($\forall$x $\in$ S)[x $\cup$ \{x\} $\in$    S]]
	\end{tcolorbox}
\end{center}
\paragraph{}在$x^+ = x \cup \{x\}$中,我们把$x^+$读作x的后续。用数学语言读无穷公理:存在一个集合,空集是这个集合的元素,而且这个集合的元素的后续也是这个集合的元素。也就是从这个无穷公理开始,数学归纳法才真正合法。思考数学归纳法的步骤:命题的首项满足,命题的第n项满足而且命题的n项的后续也满足,那么我们说命题$\forall$项满足。









